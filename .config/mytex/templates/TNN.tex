%%%%%%%%%%%%%%%%%%%%%%%%%%%%%%%%%%%%%%%%%%%%%%%%%%%%%
%   El presente documento es un machote para la creación de informes de laboratorio de física, pero
% puede modificarse para cualquier otro propósito. Está dirigido a estudiantes de la Universidad 
% Fidélitas. 
%   Este documento usa el formato para documentos de APA de la 7a edición, con la clase apa7.
% Por cualquier duda o más información puede revisar la documentación en: 
% https://www.ctan.org/pkg/apa7
%
%%%%%%%%%%%%%%%%%%%%%%%%%%%%%%%%%%%%%%%%%%%%%%%%%%%%%
\documentclass[
    aspectratio=169, %1610, 169, 149, 54, 43 and 32
    % 11pt            %Requires that the package extsize: [8pt,9pt,10pt,14pt,17pt(same Impress, PowerPoint),20pt] \ [smaller,bigger] Same as the [10pt,12pt] option. \ 12pt 
    ]{beamer}

%%%%%%%%%%%%%%%%%%%%%%%%%%%%%%%%%%%%%%%%%%%%%%%%%%%%%
%==<Importación de paquetes>=========================

%=====> Layout <=====================================

% Establece márgenes y tamaños
% \usepackage[
%     letterpaper, 
%     top=1.5cm, 
%     bottom=1.5cm, 
%     left=1.905cm, 
%     right=1.905cm
%     ]{geometry}
\interdisplaylinepenalty=2500
% Se puede establecer manualmente le indexado de la primera linea, si no se agrega el siguiente comando se usa el indexado default
%\setlength{\parindent}{4em}
% Se puede establecer el entre los párrafos, las opciones más comunes son:
%	Sin Espaciado: Comente el siguiente comentario
%	Espaciada sencillo: 1em
%	Espaciada uno y medio: 1.3em
%	Espaciada doble: 1.6 em
%\setlength{\parskip}{1em}
% Se puede establecer el espacio entre las líneas del párrafo, si no se agrega el siguiente comando se usa el espaciado default
%\renewcommand{\baselinestretch}{1.5}

%====> Definición del idioma del documento <=========

% % Paquete que cambia títulos y palabras claves al idioma seleccionado
% \usepackage[spanish]{babel}

%====> Fonts <=======================================

\usepackage{lmodern}
\usepackage{verbatim}
% Paquete que establce el idioma del documento que se va a imprimir
\usepackage{polyglossia}
\setmainlanguage{spanish}
% Paquete que carga automática los paquetres: fontspec (Paquete que permite abrir OpenType font), realscripts, metalogo .
\usepackage{xltxtra}
\setmainfont{Times New Roman}
% \setmainfont{Arial}
% \setmainfont{MathJax_Script}
%\setmonofont{Lucida Console}

% Comando que permite que el texto escrito aquí se transfiera de forma fiel al PDF
%\defaultfontfeatures{Mapping=tex-text}
%\newfontfamily\VladScrip{Vladimir Script}
\newfontfamily\arialFont{Arial}
\newfontfamily\timesnrFont{Times New Roman}

% Puedo definir cuantos comandos como quiera
%\newfontfamily\Chiller{Chiller}

% Paquete que se asegura que las letras acentuadas se acentúen correctamente
% !!! Compilar com XeTeX o LuaTeX !!!
\usepackage{xunicode}

%====> Uso de colores <===============================

% \usepackage{color}
\usepackage{xcolor}
\definecolor{BlueST} {HTML}{00e359}
\definecolor{GreenT} {HTML}{709b16}


%====> Personalización de los hipervínculos <=========

% más información en https://www.ctan.org/pkg/hyperref
\usepackage{hyperref}
\hypersetup{
    colorlinks=true,
    linkcolor=anti-flashwhite
}

% Paquete que permite incluir direcciones web de forma segura
% mas información en https://www.ctan.org/pkg/url
\usepackage{url}

%====> Letras y símbolos matemáticos <================

% Paquete que mejora la presentación visual de las ecuaciones, proporciona
% entornos adicionales para manejar varias ecuaciones o varias líneas 
% más información en https://www.ctan.org/pkg/amsmath
\usepackage{amsmath}

% Permite el uso de símbolos y formatos de letras adicionales
% como por ejemplo \mathfrak{letras}
\usepackage{amsfonts}

% Permite el uso de símbolos y formatos de letras adicionales
% más información en https://mirrors.ucr.ac.cr/CTAN/fonts/amsfonts/doc/amssymb.pdf
\usepackage{amssymb}

% Define el SI, pero tiene funciones adicionales
% más información en https://www.ctan.org/pkg/siunitx
\usepackage{siunitx}

% Negrita en cualquier texto matemático, mejor que la opción  \boldsymbol de ams
\usepackage{bm}

% Permite el uso de símbolos y formatos de letras adicionales
% como por ejemplo \mathscr{letras}
\usepackage{mathrsfs}

% Permite el uso de comandos que introducen símbolos y entornos muy comunes en las
% ecuaciones propias de la física, simplificando su escritura
\usepackage{physics}

% Para tachar en ecuaciones
\usepackage[
    makeroom,
    ]{cancel}
\renewcommand\CancelColor{\color{red}}

%====> Enumeraciones <===============================

% Permite personalizar las enumeraciones
\usepackage{enumerate}

%====> Tablas <======================================

% Permite unir celdas de varias filas 
% más información en https://www.ctan.org/pkg/multirow
\usepackage{multirow}

% Usted también puede definir su propios comandos
% Este comando permite generar rápidamente una celda que une varias 
% filas y que permite usar \\ como cambio de línea en la celda
\newcommand{\minitab}[2][]{\begin{tabular}{#1}#2\end{tabular}}

% Permite usar columnas de ancho preestablecido
\usepackage{dcolumn}
\newcolumntype{L}[1]{>{\raggedright\let\newline\\\arraybackslash\hspace{0pt}}m{#1}}
\newcolumntype{C}[1]{>{\centering\let\newline\\\arraybackslash\hspace{0pt}}m{#1}}
\newcolumntype{R}[1]{>{\raggedleft\let\newline\\\arraybackslash\hspace{0pt}}m{#1}}

%====> Imágenes <====================================

\usepackage{subcaption}
\usepackage{graphicx}
\usepackage{graphics}

% Permite importar directamente los svg y procesar el texto con latex
\usepackage{svg}
\usepackage[off]{svg-extract}
\svgsetup{clean=true,inkscapearea=page}
\usepackage{relsize}

\usepackage{tikz}
\usetikzlibrary{patterns,angles}
\usetikzlibrary{calc,arrows,shapes.arrows,intersections,math}
\usetikzlibrary{decorations,decorations.markings,babel,positioning}
%====> Formato de la pagina <========================

% Permite dividir la hoja en varias columnas
% más información en https://www.ctan.org/pkg/multicol
\usepackage{multicol}

%====> Gestor de bibliografía <======================

% más información en https://www.ctan.org/pkg/biblatex
\usepackage[
    backend=biber, 
    % style=apa, 
    % citestyle=apa
    ]{biblatex}

% Paquete auxiliar para las citas
\usepackage{csquotes}

% \usepackage{import}
% \usepackage{pdfpages}
% \usepackage{transparent}
% \usepackage{xcolor}

% \newcommand{\incfig}[2][1]{%
%     \def\svgwidth{#1\columnwidth}
%     \import{./figures/}{#2.pdf_tex}
% }
% \newcommand{\incsvg}[2][1]{
% 	\includesvg[
% 				height = #1\textheight,
% 				% width=\linewidth,
% 				pretex=\relscale{1.5}
% 			]{#2}
% }
% \pdfsuppresswarningpagegroup=1

%%%%%%%%%%%%%%%%%%%%%%%%%%%%%%%%%%%%%%%%%%%%%%%%%%%%%
%==<Definición de portada y encabezados>=============

% \usepackage{fancyhdr}

% %=====> Define header <==============================
% \pagestyle{fancy}
% \fancyhf{}
% \lhead{\textit{Luis Fernando Umaña Castro}}
% \rhead{\textit{Soluciones - Semana 8}}
% \cfoot{\thepage}

% %====> Espacio para el encabezado <==================
% \setlength{\headheight}{15pt}
% \addtolength{\topmargin}{-0.15pt}
\usetheme{CambridgeUS}
% \usetheme{Nord}
\usecolortheme{whale}
% \usecolortheme{Nord}
\definecolor{coolblack}{rgb}{0.0, 0.18, 0.39}
\definecolor{anti-flashwhite}{rgb}{0.95, 0.95, 0.96}

\setbeamercolor{frametitle}{fg=coolblack}
\setbeamercolor{title}{fg=anti-flashwhite}
\setbeamercolor{title in head/foot}{fg=anti-flashwhite}
\setbeamercolor{title in sidebar}{fg=anti-flashwhite}
% \usefonttheme{UFide}
\usefonttheme[onlymath]{serif}
\title[Unidades y Vectores]{Semana 1 \\ Física, medición y vectores.}
% The short title appears at the bottom of every slide, the full title is only on the title page

\author{Luis Fernando Umaña Castro} 
% Your name
\institute[UCR] 
% Your institution as it will appear on the bottom of every slide, may be shorthand to save space
{
Universidad de Costa Rica \\ % Your institution for the title page
\medskip
\textit{luis.umanacastro@ucr.ac.cr} \\ % Your email address\\ % Your institution for the title page
\medskip
Oficina 430FM \\ Consulta: J(9:00 a 12:00)  % Your email address
}
\date{} % Date, can be changed to a custom date

%%%%%%%%%%%%%%%%%%%%%%%%%%%%%%%%%%%%%%%%%%%%%%%%%%%%%
%==<Definición de comandos propios>=============
\newcommand{\TCfolder}{/home/luis/.config/mytex}
\newcommand{\AAfolder}{/home/luis/Documents/00-00AA-Apuntes}
% \newcommand{\AAfolder}{/mnt/c/MyFiles/Documents/02-U/01-Fisica/00-00AA-Apuntes}
\newcommand{\scrp}[1]{\mbox{\scriptsize{#1}}}
\newcommand{\vc}[1]{\vec{\bm{{#1}}}}
\newcommand{\vm}[1]{\bm{{#1}}}
\newcommand{\vh}[1]{\hat{\bm{{#1}}}}
\newcommand{\ala}[1]{$^{#1}$}
% \renewcommand{\sin}{\mbox{ sen}}
\let\oldunit\unit
\renewcommand{\unit}[1]{\:\oldunit{#1}}
\newcommand{\then}{\pause =}
% % En el siguiente comando se indica cuál es el archivo que tiene las referencias
\usepackage{\TCfolder/sty/ColorsLight}
\usepackage{\TCfolder/sty/SetSymbols}
\addbibresource{\TCfolder/bib/Biblioteca.bib}
%%%%%%%%%%%%%%%%%%%%%%%%%%%%%%%%%%%%%%%%%%%%%%%%%%%%%
%==<Inicia el documento>=============================
\begin{document} %%%%%%%%%%%%%%%%%%%%%%%%%%%%%%%%%%%%
%%%%%%%%%%%%%%%%%%%%%%%%%%%%%%%%%%%%%%%%%%%%%%%%%%%%%
\maketitle
%%%%%%%%%%%%%%%%%%%%%%%%%%%%%%%%%%%%%%%%%%%%%%%%%%%%%
%==<Frame>===========================================
\section{Unidades}
\begin{frame}\frametitle{Unidades}\pause
    Sistema internacional de Unidades\pause
    \begin{center}
        \includegraphics[height = 0.7\textheight]{\AAfolder/img/10MC/NIST/C1S1-img-001}
    \end{center}
\end{frame}
%==<Frame>===========================================
\subsection{Análisis Dimensional}
\begin{frame}
    \frametitle{Análisis Dimensional} \pause
    Podemos manipular los símbolos de las unidades con las reglas del álgebra \pause (leyes de los exponentes) \pause
    \[\frac{\oldunit{m}}{\oldunit{s}^2} \cdot \oldunit{s} \then \oldunit{m . s^{-2} . s^{1}} \then  \oldunit{m . s^{-2+1} \then m . s^{-1} \then \frac{\oldunit{m}}{\oldunit{s}} } \] \pause
    Es una buena práctica siempre revisar las unidades de la fórmula resultante en un despeje matemático\pause. Por ejemplo, analicemos si la siguiente fórmula tiene las unidades adecuadas de la velocidad $\unit{m/s}$
    \[v = \sqrt{2gh}\]\pause
    \[(\oldunit{m/s^2}\cdot\oldunit{m})^{1/2} \then (\oldunit{m^2/s^2})^{1/2} \then \oldunit{m/s} \]
\end{frame}
%==<Frame>===========================================
\subsection{Conversión de unidades}
\begin{frame}
    \frametitle{Conversión de unidades} \pause
    Se necesita una identidad entre unidades diferentes\pause, por ejemplo:
    \[1\unit{in} = 2.54\unit{cm}\]\pause
    A partir de la identidad se crea el factor de conversión\pause, por ejemplo:
    \[18\unit{in}\times\underset{\minitab[c]{Factor de\\ conversión}}{\left(\frac{2.54\unit{cm}}{1\unit{in}}\right)} \then 45.72\unit{cm} \] \pause
    El factor de conversión se construye de manera tal que se cancelen las unidades que se desean eliminar.
\end{frame}
%==<Frame>===========================================
\begin{frame}
    \frametitle{Ejemplo} \framesubtitle{Sistema imperial o inglés}\pause
    Convertir 1 milla a kilómetros \pause
    \[1 \unit{mi} \pause\times\frac{8 \unit{fur}}{1 \unit{mi}}\pause\times\frac{10 \unit{ch}}{1 \unit{fur}}\pause\times\frac{4 \unit{rod}}{1 \unit{ch}}\pause\times\frac{5.5 \unit{yd}}{1 \unit{rod}}\pause\times\frac{3 \unit{ft}}{1 \unit{yd}}\pause\times\frac{12 \unit{in}}{1 \unit{ft}}\pause\times\frac{2.54 \unit{cm}}{1 \unit{in}}\pause\times\frac{1 \unit{m}}{100 \unit{cm}}\pause\times\frac{1 \unit{km}}{1000 \unit{m}}\]\pause
    \[1 \unit{mi} \times\frac{8 \unit{fur}}{1 \unit{mi}}\times\frac{10 \unit{ch}}{1 \unit{fur}}\times\frac{4 \unit{rod}}{1 \unit{ch}}\times\frac{5.5 \unit{yd}}{1 \unit{rod}}\times\frac{3 \unit{ft}}{1 \unit{yd}}\times\frac{12 \unit{in}}{1 \unit{ft}}\times\frac{2.54 \unit{cm}}{1 \unit{in}}\times\frac{1 \unit{m}}{10^2 \unit{cm}}\times\frac{1 \unit{km}}{10^3 \unit{m}}\]\pause
    \[1 \unit{mi} \times\frac{8 \unit{fur}}{1 \unit{mi}}\times\frac{10 \unit{ch}}{1 \unit{fur}}\times\frac{4 \unit{rod}}{1 \unit{ch}}\times\frac{5.5 \unit{yd}}{1 \unit{rod}}\times\frac{3 \unit{ft}}{1 \unit{yd}}\times\frac{12 \unit{in}}{1 \unit{ft}}\times\frac{2.54 \unit{cm}}{1 \unit{in}}\times\frac{1 \unit{km}}{10^5 \unit{cm}}\]\pause
    \[\frac {1 \times 8 \times 10 \times 4 \times 5.5 \times 3 \times 12 \times 2.54}{10^5} \then 1.609344 \unit{km}\]
\end{frame}
%==<Frame>===========================================
\begin{frame}
    \frametitle{Ejemplo} \framesubtitle{Unidades con exponentes}\pause
    Convertir $1 \unit{cm^3}$ a $\unit{m^3}$ \pause
    \[1\unit{cm^3} \pause\times \frac{1 \unit{m}}{100 \unit{cm}} \then 0.01 \unit{cm^2 m}\]\pause
    \[1\unit{cm^3} \times \left(\frac{1 \unit{m}}{100 \unit{cm}}\right)^3 \then 1\unit{cm^3} \times \frac{1 \unit{m^3}}{100^3 \unit{cm}^3} \then 10^{-6} \unit{m^3}\]
\end{frame}
%==<Frame>===========================================
\subsection{Múltiplos y Submúltiplos}
\begin{frame}
    \frametitle{Múltiplos y Submúltiplos}\pause
    \begin{center}
        \includegraphics[width = \textwidth]{\AAfolder/img/10MC/530_S492fi10-Serway/tab01001.png}
    \end{center}
\end{frame}
%%%%%%%%%%%%%%%%%%%%%%%%%%%%%%%%%%%%%%%%%%%%%%%%%%%%%
%==<Frame>===========================================
\section{Cantidades escalares y vectoriales.}
\begin{frame}
	\frametitle{Definiciones} \pause
    \begin{description}
        \item[Escalar] Cantidad con magnitud.\pause
        \item[Vector] Cantidad con magnitud y dirección.\pause
    \end{description}
\end{frame}
%%%%%%%%%%%%%%%%%%%%%%%%%%%%%%%%%%%%%%%%%%%%%%%%%%%%%
%==<Frame>===========================================
\subsection{Vector}
\begin{frame}
	\frametitle{Naturaleza del vector}\pause
    \begin{itemize}
        \item Se representa en letras negritas $\bm{A}$ o con una flecha sobre la letra $\vec{A}$.\pause
        \item Gráficamente se dibuja con una flecha.\pause \\
        \begin{center}
            \includesvg[
				height = 0.1\textheight,
				% width=\linewidth,
				pretex=\relscale{1}
			]{img/vecA}
        \end{center}\pause
		\item Dos vectores son paralelos si tienen la misma dirección, y son iguales si además tiene la misma magnitud.\pause \\
    \end{itemize}
	\begin{center}
		\includesvg[
			height = 0.2\textheight,
			% width=\linewidth,
			pretex=\relscale{1}
		]{img/vecA-vecB}
	\end{center}
\end{frame}
%%%%%%%%%%%%%%%%%%%%%%%%%%%%%%%%%%%%%%%%%%%%%%%%%%%%%
%==<Frame>===========================================
\subsection{Vector}
\begin{frame}
	\frametitle{Naturaleza del vector}\pause
	\begin{itemize}
        \item Un vector negativo significa que tiene la misma magnitud que el vector original pero dirección opuesta.\pause
        \begin{center}
            \includesvg[
				height = 0.2\textheight,
				% width=\linewidth,
				pretex=\relscale{1}
			]{img/vecBasNegVecA}
        \end{center}\pause
        \item La magnitud de un vector se puede representar de la forma: Magnitud de $\vc{A} = A = |\vc{A}|$.\pause
        \item La magnitud es un escalar y siempre es positiva.
    \end{itemize}
\end{frame}
%%%%%%%%%%%%%%%%%%%%%%%%%%%%%%%%%%%%%%%%%%%%%%%%%%%%%
%==<Frame>===========================================
\section{Componentes de un vector y vectores unitarios.}
\begin{frame}
	\frametitle{Componentes del vector}\pause
	\begin{itemize}
		\item Establecer un sistema rectangular de ejes de coordenadas.\pause
        \begin{center}
            \includesvg[
				height = 0.2\textheight,
				% width=\linewidth,
				pretex=\relscale{1}
			]{img/cartCoord}
        \end{center}\pause
		\item Colocar el vector con el inicio en el origen $\mathcal{O}$.\pause
        \begin{center}
            \includesvg[
				height = 0.2\textheight,
				% width=\linewidth,
				pretex=\relscale{1}
			]{img/cartCoordVecA}
        \end{center}
	\end{itemize}
\end{frame}
%==<Frame>===========================================
\begin{frame}
	\frametitle{Componentes del vector}\pause
	\begin{itemize}
		\item Distinguir las proyecciones del vector sobre cada eje ($\vAx$ y ${A}_y$).\pause
        \begin{center}
            \includesvg[
				height = 0.2\textheight,
				% width=\linewidth,
				pretex=\relscale{1}
			]{img/cartCoordVecAComp}
        \end{center}\pause
		\item Las componentes son las magnitudes de dichas proyecciones, son escalares.\pause
		\item Las componentes junto con el vector forman un triangulo rectángulo.\pause
        \begin{center}
            \includesvg[
				height = 0.2\textheight,
				% width=\linewidth,
				pretex=\relscale{1}
			]{img/vecACompTriang}
        \end{center}\pause
		La magnitud del vector es la hipotenusa del triángulo, y las componentes los catetos.
	\end{itemize}
\end{frame}
%==<Frame>===========================================
\begin{frame}
	\frametitle{Componentes del vector}\pause

    \begin{columns}
		\begin{column}{0.5\textwidth}
    \centering%
    \textbf{Coordenadas Cartesianas}
%========================================================================
    \begin{tikzpicture}[x=2cm,y=2cm,line width=1.25pt,cap=round,>=latex]
% Definitions------------------------------------------------------------
        \tikzmath{
        \xVal = sqrt(3)/2;
        \yVal = 0.5;
        }
        \coordinate(o) at (0,0);
        \coordinate(p) at (30:1);
        \coordinate(px) at (\xVal,0);
        \coordinate(py) at (0,\yVal);
        
        \path (-0.2,-0.2) -- (1.2,1.2);
% Axis-------------------------------------------------------------------
        \draw[->, draw = PosX] (-0.1,0) -- (1,0) node[anchor = west] {$\vi$};
        \draw[->, draw = PosY] (0,-0.1) -- (0,1) node[anchor = south] {$\vj$};
% Polar components-------------------------------------------------------
%Diffused----------------------------------------------------------------
        \draw[->,draw = black!60!white] (o) -- (p) node[anchor=south west,at end] {$\vA = (\vAx,\vAy)$};
            
% Cartesian components---------------------------------------------------
%Solid-------------------------------------------------------------------
        \draw[dashed,draw = PosX] (p) -- (px) node[anchor = north] {$\vAx$};
        \draw[dashed,draw = PosY] (p) -- (py) node[anchor = east] {$\vAy$};
    \end{tikzpicture}
%========================================================================
    \end{column} \pause
    \begin{column}{0.5\textwidth}
    \centering
    \textbf{Coordenadas Polares} \newline
%========================================================================
\begin{tikzpicture}[x=2cm,y=2cm,line width=1.25pt,cap=round,>=latex]
% Definitions------------------------------------------------------------
    \tikzmath{
    \xVal = sqrt(3)/2;
    \yVal = 0.5;
    }
    \coordinate(o) at (0,0);
    \coordinate(p) at (30:1);
    \coordinate(px) at (\xVal,0);
    \coordinate(py) at (0,\yVal);
    
    \path (-0.2,-0.2) -- (1.2,1.2);
    
% Axis-------------------------------------------------------------------
    \draw[->, draw = PosX] (0,0) -- (1,0) node[anchor = north west] {$\vi$};
    
% Polar components-------------------------------------------------------
%Solid-------------------------------------------------------------------
    \draw[->,draw = black] (o) -- (p) node[anchor=south west, at end] {$\vA = (\mvA,\vAt)$};
    \draw[draw = black!75!white] (0.5,0) arc [start angle=0, end angle=30, radius=0.5] node[anchor=west, midway] {$\vAt$};;
\end{tikzpicture}
%========================================================================

\end{column}
\end{columns}
\end{frame}
%==<Frame>===========================================
\begin{frame}
	\frametitle{Cálculo de las componentes a partir de magnitud y dirección del vector}\pause
	\begin{itemize}
		\item Se usan las definiciones de las funciones trigonométricas.\pause
		\item La dirección del vector se representa como el ángulo $\vAt$ que forma el vector con respecto a un eje.\pause
		\newline Para asignar el seno o el coseno es importante tener bien claro cuál es el eje desde el cual se está asignando el ángulo.\pause
		\newline \textbf{Seno} Situándose en el ángulo pregunto ¿cuál cateto está en frente al ángulo?, la magnitud de ese cateto es $\mvA\sin\vAt$.\pause
		\newline \textbf{Coseno} Situándose en el ángulo pregunto ¿cuál cateto está a la par del ángulo?, la magnitud de ese cateto es $\mvA\cos\vAt$.\pause
	\end{itemize}\begin{center}
%========================================================================
	\begin{tikzpicture}[x=2cm,y=2cm,line width=1.25pt,cap=round,>=latex]
			% Definitions------------------------------------------------------------
				\tikzmath{
				\xVal = sqrt(3)/2;
				\yVal = 0.5;
				}
				\coordinate(o) at (0,0);
				\coordinate(p) at (30:1);
				\coordinate(px) at (\xVal,0);
				\coordinate(py) at (0,\yVal);
				
			% Axis-------------------------------------------------------------------
				\draw[draw = PosX] (0,0) -- (\xVal,0) node[anchor = north, midway] {$\vAx = \mvA\cos\vAt $};
				\draw[draw = PosY] (\xVal,0) -- (\xVal,\yVal) node[anchor = west, midway] {$\vAy = \mvA\sin\vAt$};
				
			% Polar components-------------------------------------------------------
			%Solid-------------------------------------------------------------------
				\draw[->,draw = Pos!75!red] (o) -- (p) node[anchor=south east, midway] {$\mvA$};
				\draw[draw = Pos!60!red] (0.5,0) arc [start angle=0, end angle=30, radius=0.5] node[anchor=west, midway] {$\vAt$};;
		\end{tikzpicture}
		%========================================================================
		\hspace{3cm}
		%========================================================================
		\begin{tikzpicture}[x=2cm,y=2cm,line width=1.25pt,cap=round,>=latex]
			% Definitions------------------------------------------------------------
				\tikzmath{
				\xVal = sqrt(3)/2;
				\yVal = 0.5;
				}
				\coordinate(o) at (0,0);
				\coordinate(p) at (30:1);
				\coordinate(px) at (\xVal,0);
				\coordinate(py) at (0,\yVal);
				
			% Axis-------------------------------------------------------------------
				\draw[draw = PosX] (0,\yVal) -- (\xVal,\yVal) node[anchor = south, midway] {$\vAx = \mvA\sin\tAB $};
				\draw[draw = PosY] (0,0) -- (0,\yVal) node[anchor = east, midway] {$\vAy = \mvA\cos\tAB$};
				
			% Polar components-------------------------------------------------------
			%Solid-------------------------------------------------------------------
				\draw[->,draw = Pos!75!red] (o) -- (p) node[anchor=north west, midway] {$\mvA$};
				\draw[draw = blue] (30:0.25) arc [start angle=30, end angle=90, radius=0.25] node[anchor=south west, midway] {$\tAB$};;
		\end{tikzpicture}
		%========================================================================
	\end{center}
\end{frame}
%==<Frame>===========================================
\begin{frame}
	\frametitle{Cálculo de la magnitud y dirección del vector a partir de las componentes del vector}\pause
	\begin{itemize}
		\item \textbf{Magnitud} Se usa el teorema de Pitágoras.
		\begin{equation}
			\mvA = \sqrt{\vAx^2 + \vAy^2}
			\label{eq:10MC:C1S2:Pitagoras}
		\end{equation}\pause
		\item \textbf{Dirección} Se utiliza la definición del tangente $\displaystyle\tan\vAt = \frac{\vAy}{\vAx}$.\pause
		\begin{equation}
			\vAt = \arctan(\frac{\vAy}{\vAx})
			\label{eq:10MC:C1S2:CalcAng}
		\end{equation}\pause
		\newline En general es necesario tener cuidado al momento de indicar la dirección, ya que esta función tiene dos soluciones.\pause
		% \newline Las calculadoras suelen dar el valor el ángulo que forma parte del triángulo rectángulo.\pause
		% \newline Según se definió en \eqref{eq:10MC:C1S2:CalcAng} la calculadora daría el ángulo entre el eje que contenga a $\vAx$ y la hipotenusa.
	\end{itemize}
\end{frame}
%==<Frame>===========================================
\begin{frame}
	\frametitle{Vectores unitarios}\pause
	\begin{itemize}
		\item Son vectores con magnitud igual a 1.\pause
		\item Su finalidad es describir una dirección.\pause
		\item Se reconocen por usar el acento circunflejo $\vh{a}$ en lugar de la flecha o simplemente la negrita.\pause
		\item Consenso:
		\begin{itemize}
			\item La dirección positiva del eje x se describe por $\vh{i}$.
			\item La dirección positiva del eje y se describe por $\vh{j}$.
			\item La dirección positiva del eje z se describe por $\vh{k}$.
		\end{itemize}
	\end{itemize}
\end{frame}
%%%%%%%%%%%%%%%%%%%%%%%%%%%%%%%%%%%%%%%%%%%%%%%%%%%%%
%==<Frame>===========================================
\section{Operaciones con vectores}
\begin{frame}
	\frametitle{Suma de vectores}\pause
	\begin{itemize}
		\item La suma vectorial es una operación geométrica.\pause
	\end{itemize}
	\begin{figure}[!ht]
		\centering
		\includegraphics[width=\textwidth]{\AAfolder/img/10MC/530_S439fi12-Sears/C1S2-img-001.png} 
		\caption[caption]{Suma de vectores. \hfill {\footnotesize {Fuente:} \cite{Sears1}}}
		\label{fg:10MC:C1S2:SumaVec}
	\end{figure}
	\begin{itemize}
		\item Es conmutativa: El orden de los vectores no altera el resultado.\pause
		\item Es asociativa: Si se tienen tres vectores se pueden sumar primero dos y luego el tercero.
	\end{itemize}
\end{frame}
%==<Frame>===========================================
\begin{frame}
	\frametitle{Importancia de la suma vectorial}\pause
	\begin{itemize}
		\item Todo vector se puede representar como la suma vectorial de las dos proyecciones que genera $\vc{A} = \vc{A}_x + \vc{A}_y$.\pause
		\item Cualquier vector se puede escribir como la suma de sus componentes, cada una multiplicando el vector unitario que marca la dirección sobre la cual se calculó la componente\pause
		\begin{equation}
			\vA = \vAx\vh{i} + \vAy\vh{j} + \vAz\vh{k}
			\label{eq:10EM:VecUnit}
		\end{equation}\pause
		\item Esta notación facilita el álgebra vectorial.
	\end{itemize}
\end{frame}
%==<Frame>===========================================
\begin{frame}
	\frametitle{Suma usando componentes de vectores} \pause
	Sea \[\vA = \vAx \vh{i} + \vAy \vh{j} + \vAz \vh{k} \] \pause y \[\vB = \vBx \vh{i} + \vBy \vh{j} + \vBz \vh{k} \] \pause
	% Se tiene que tomar todas las componentes que correspondan a una misma dirección. Se suman todas ellas, el resultado es el valor de la componente en la dirección seleccionada del vector resultante.
	Procedemos entonces a realizar la suma
	\[\vA + \vB = (\vAx\vi + \vAy\vj + \vAz\vk) + (\vBx\vi + \vBy\vj + \vBz\vk)\]
	\[\vA + \vB = \vAx\vi + \vBx\vi + \vAy\vj + \vBy\vj + \vAz\vk + \vBz\vk\]
	\[\vA + \vB = (\vAx + \vBx)\vi + (\vAy + \vBy)\vj + (\vAz + \vBz)\vk\]
	Se puede generalizar el proceso diciendo que se suman toas las componentes que vayan en la misma dirección, sin importar la cantidad de vectores.
	% \begin{itemize}
	% 	\item Se tiene que tomar todas las componentes que correspondan a una misma dirección. 
	% 	\item Se suman todas ellas, el resultado es el valor de la componente en la dirección seleccionada del vector resultante.
	% 	\item Esto se debe repetir con cada dirección para obtener todas las componentes del vector resultante.
	% \end{itemize}
\end{frame}
%==<Frame>===========================================
\begin{frame}
	\frametitle{Multiplicación con un escalar}
	\par Al multiplicar un vector por un escalar, se multiplica la magnitud del vector y se mantiene la dirección.\pause Esto es completamente equivalente a multiplicar cada una de las componentes por el mismo escalar \pause
	\[c\vA \then (c\mvA, \vAt) \then (c\vAx,c\vAy,c\vAz)\]\pause
	\begin{center}
		\includesvg[
			height = 0.2\textheight,
			% width=\linewidth,
			pretex=\relscale{1}
		]{img/2vecA}
	\end{center}
\end{frame}
%==<Frame>===========================================
\begin{frame}
	\frametitle{Producto punto (escalar)}
	\begin{itemize}\pause
		\item Se llama producto escalar, porque el resultado de este producto entre dos vectores es un escalar\pause
		\item Se denota con un punto $\vA\cdot\vB$.\pause
		\item Es Conmutativo\pause
		\item Se define el ángulo $\tAB$ como el ángulo entro los dos vectores (al colocar el inicio de ambos en un mismo punto) \pause
		\begin{equation}
			\vA\cdot\vB = \mvA\mvB\cos\tAB
			\label{eq:10MV:DefProdPunto}
		\end{equation}\pause
		\item El \textbf{producto escalar} de dos vectores \textbf{perpendiculares} es \textbf{siempre} cero.
	\end{itemize}
\end{frame}
%==<Frame>===========================================
\begin{frame}
	\frametitle{Producto punto (escalar)}
	Usando componentes \pause
	\[\mqty{\vA\cdot\vB &=&  (\vAx\vi+\vAy\vj+\vAz\vk)\cdot(\vBx\vi+\vBy\vj+\vBz\vk) }\]\pause
	note que en el lado derecho podemos distribuir el producto término por término:\pause
	\[\mqty{\vA\cdot\vB 
	&=& \vAx\vBx(\vi\cdot\vi) + \vAx\vBy(\vi\cdot\vj) + \vAx\vBz(\vi\cdot\vk) \\\pause
	&& +\vAy\vBx(\vj\cdot\vi) + \vAy\vBy(\vj\cdot\vj) + \vAy\vBz(\vj\cdot\vk) \\\pause
	&& +\vAz\vBx(\vk\cdot\vi) + \vAz\vBy(\vk\cdot\vj) + \vAz\vBz(\vk\cdot\vk) }\]\pause
	considere que
	\[\vi\cdot\vi = \vj\cdot\vj =\vk\cdot\vk \then 1\cdot 1 \cos 0 \then 1\] \pause
	\[\vi\cdot\vj = \vj\cdot\vk =\vk\cdot\vi \then 1\cdot 1 \cos 90^{\circ} \then 0\] \pause
		\begin{equation}
			\vc{A}\cdot\vc{B} = \vAx\vBx+\vAy\vBy+\vAz\vBz
			\label{eq:10MC:C1S2:ProdPuntComp}
		\end{equation}
\end{frame}
%==<Frame>===========================================
\begin{frame}
	\frametitle{Producto cruz (vectorial)}
	\begin{itemize}
		\item El resultado de un producto cruz es un vector. \pause
		\item El vector resultante es perpendicular simultáneamente a ambos vectores. \pause
		\item Se designa con una equis $\vA\times\vB= \vc{C}$. \pause
		\item La \textbf{magnitud} del vector resultante se puede calcular de la forma
		\begin{equation}
		C = \mvA\mvB\sin\tAB
		\label{eq:10MC:C1S2:DefMagProdCruz}
		\end{equation} \pause
		\item La dirección positiva del vector resultante se designa por la regla de la mano derecha\pause
		\newline Usando la mano \textbf{derecha} se extienden los dedo en dirección del primer vector, luego se cierra el puño en dirección al segundo vector. La dirección que tenga el pulgar es la dirección positiva.\pause
		\item El producto cruz \textbf{NO} es conmutativo.\pause
		\item El \textbf{producto cruz} de dos vectores \textbf{paralelos} es \textbf{siempre} cero.
	\end{itemize}
\end{frame}
%==<Frame>===========================================
\begin{frame}
	\frametitle{Producto cruz (vectorial)}\pause
	Usando componentes\pause
	\[\mqty{\vA\times\vB &=&  (\vAx\vi+\vAy\vj+\vAz\vk)\times(\vBx\vi+\vBy\vj+\vBz\vk) \\\pause
	&=& \vAx\vBx\vi\times\vi + \vAx\vBy\vi\times\vj + \vAx\vBz\vi\times\vk \\ \pause
	&& +\vAy\vBx\vj\times\vi + \vAy\vBy\vj\times\vj + \vAy\vBz\vj\times\vk \\\pause
	&& +\vAz\vBx\vk\times\vi + \vAz\vBy\vk\times\vj + \vAz\vBz\vk\times\vk }\]\pause
	considere que\pause
	\[\vi\times\vi = \vj\times\vj =\vk\times\vk = 0\]\pause
	\par En cuanto a direcciones diferentes siguen el siguiente orden:\pause
	\[\mqty{\vi\times\vj = \vk \pause&\hspace{0.5cm}& \vj\times\vi = -\vk \\\pause
			\vj\times\vk = \vi \pause&\hspace{0.5cm}& \vk\times\vj = -\vi \\\pause
			\vk\times\vi = \vj \pause&\hspace{0.5cm}& \vi\times\vk = -\vj}\]
\end{frame}
%==<Frame>===========================================
\begin{frame}
	\frametitle{Producto cruz (vectorial)}\pause
	Agrupando en el producto vectorial y sustituyendo los productos entre direcciones se obtiene que:\pause
		\begin{equation}
            \vA\times\vB = \left(\vAy\vBz - \vAz\vBy\right)\vh{i} + \left(\vAz\vBx - \vAx\vBz\right)\vh{j} + \left(\vAy\vBx - \vAx\vBy\right)\vh{k}
            \label{eq:10MC:C1S2:ProdCruzComp}
		\end{equation}\pause
	\begin{itemize}
		\item También se puede calcular por medio de la notación matricial del determinante\pause
	\end{itemize} 
		\begin{equation}
            \vA\times\vB = \mqty|\vh{i}&\vh{j}&\vh{k}\\\vAx&\vAy&\vAz\\\vBx&\vBy&\vBz|
            \label{eq:10MC:C1S2:ProdCruzDet}
		\end{equation}
\end{frame}
%%%%%%%%%%%%%%%%%%%%%%%%%%%%%%%%%%%%%%%%%%%%%%%%%%%%%
%==<Bibliografía>====================================
%   Para la bibliografía se usa el gestor de referencias bibtex. Toda la información de las referencias
% se pone en un archivo aparte con la terminación .bib y se agrega aquí
%
%==<Frame>===========================================
\begin{frame}
    \printbibliography
\end{frame}
%%%%%%%%%%%%%%%%%%%%%%%%%%%%%%%%%%%%%%%%%%%%%%%%%%%%%
%==<Apéndice>========================================
%   Se puede crear apéndices que luego se pueden referenciar con la etiqueta
% \appendix
% \section{Figuras \label{app:fig}}
%%%%%%%%%%%%%%%%%%%%%%%%%%%%%%%%%%%%%%%%%%%%%%%%%%%%%
%==<Fin del documento>===============================
\end{document}
