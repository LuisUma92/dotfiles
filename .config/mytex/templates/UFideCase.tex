\begin{center}
  {\Huge \thetitle} \\[2em]
  {\large \thecycle} % \medskip \\
\end{center}
\begin{center}
\begin{tabular}{|C{0.59\textwidth}|C{0.3\textwidth}|}
\hline
\textbf{Nambre, Apellitos} & \textbf{Cédula} \\ \hline
         & \\ \hline
\end{tabular}
\end{center}
\noindent
\textbf{La prueba es totalmente individual y se debe subir al campus virtual
  de forma obligatoria en formato PDF
  (en caso de no subirse en este formato, no se califica la prueba) }
Tiempo máximo para la resolución del estudio de caso: 2 horas y 30 minutos
\renewcommand{\labelenumi}{\textbf{\arabic{enumi}.}}
\begin{enumerate}
\item \textbf{Descripción de la actividad} \\
  Resuelva los siguientes casos presentados a continuación,
  que están relacionados con \topics.
  A la hora de presentar su resolución:
  \begin{enumerate}[a.]
    \item Deben aparecer todos los procedimientos algebraicos que le llevaron
      a la solución final, ya que si no aparecen, no se otorgan puntos.
    \item Utilice los esquemas y dibujos necesarios para la confección de
      la solución.
    \item En las partes de la solución, debe aparecer una descripción y
      justificación de los pasos realizados,
      indicando el porqué del uso de las leyes físicas determinadas,
      análisis vectorial en los casos que corresponda y
      comentarios sobre el resultado final.
    \item Debe indicar con claridad cuál es el resultado final pues,
      de lo contrario, se asumirá que no llegó al mismo.
  \end{enumerate}
\item \textbf{Formato}\\
  Se debe entregar un único archivo con la solución detallada de los casos,
  puede hacerlos a mano con letra legible y tomar una foto legible y
  pegar las fotos en un documento,
  o escanear sus soluciones y entregar este escaneo.
  O también, se pueden resolver en computadora con procesador de texto o
  con ayuda de algún digitalizador
  (stylus (lápiz para escribir sobre la pantalla) o dispositivo similar).

  Suba el caso en un documento en PDF
  (en caso de no subirse en este formato, no se califica la prueba)
\item \textbf{Entrega}\\
  La entrega de este ejercicio se debe hacer antes que finalice la clase
  en formato PDF en el campus virtual.
  Cada estudiante de forma individual debe subir su documento.
\end{enumerate}
  \begin{center}
    \vspace{1.5cm}
    \thepage
  \end{center}
  \newpage
\begin{enumerate}
  \setcounter{enumi}{3}
\item \textbf{Otras Pautas}
  \begin{itemize}
    \item
      Durante la ejecución de la prueba no se permite el uso de redes sociales,
      WhatsApp, Telegram, Line (entre otros medios electrónicos o personales)
      que se puedan utilizar para obtener una ayuda con la elaboración del
      estudio de caso.
    \item
      Es obligación del estudiante verificar que el archivo subido al campus
      virtual en formato PDF funciona perfectamente.
      Se solicita a cada estudiante descargar el documento que se subió al
      campus para verificar que el mismo abre perfectamente.
      Este proceso es una responsabilidad del estudiante.
   \item
     Esta asignación es individual,
     se les recuerda que se espera un comportamiento acorde con
     los más altos valores de integridad académica.
    \item
      En caso de que su docente determine que se ha copiado una parte o
      la totalidad de la solución de la tarea,
      la persona o personas en las que se detecte esa “similitud”
      tendrán una nota de cero.
      Seguido de una advertencia en la primera ocasión,
      si se repite se procederá con lo que corresponde a lo estipulado para 
      el comportamiento fraudulento en el reglamento de la
      Universidad Fidélitas.
  \end{itemize}
\end{enumerate}
