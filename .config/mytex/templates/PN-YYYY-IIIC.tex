% !TeX program = lualatex
% !TeX root    = main.tex
%%%%%%%%%%%%%%%%%%%%%%%%%%%%%%%%%%%%%%%%%%%%%%%%%%%%%%%%%%%%%%%%%%%%%%%%%%%%%%%
\documentclass[
  12pt,
  notitlepage,
  openany,
  twoside,
  %twocolumn,
  ]{book}
%%%%%%%%%%%%%%%%%%%%%%%%%%%%%%%%%%%%%%%%%%%%%%%%%%%%%%%%%%%%%%%%%%%%%%%%%%%%%%%
  %==< Definir paquetes >======================================================
  \usepackage{config/0_packages}
  \usepackage{config/1_loyaut}
  \usepackage{config/2_commands}
  \usepackage{config/2_partial}
  \usepackage{config/3_units}
  %==<Definir bibliografía>====================================================
  \usepackage{config/4_biber}
  %==< Definir recursos visuales >=============================================
  \graphicspath{
    {img}
    % {\FisicaDir/00II-ImagesFigures/10MC}
    % {\FisicaDir/00II-ImagesFigures/20TD}
    % {\FisicaDir/00II-ImagesFigures/40EM}
    % {\FisicaDir/00II-ImagesFigures/530_B344f1-Bauer}
    % {\FisicaDir/00II-ImagesFigures/530_G433f4-Giancoli}
    % {\FisicaDir/00II-ImagesFigures/530_R434f4-resnickF4}
    % {\FisicaDir/00II-ImagesFigures/530_R434f5-resnick}
    % {\FisicaDir/00II-ImagesFigures/530_S439fi12-Sears}
    % {\FisicaDir/00II-ImagesFigures/530_S439fi14-Sears}
    % {\FisicaDir/00II-ImagesFigures/530_S492fi10-Serway}
    % {\FisicaDir/00II-ImagesFigures/530_W696f6}
    % {\FisicaDir/00II-ImagesFigures/icons}
  }
  %==< Definir información general >===========================================
  \title{}
  \date{}
  \usepackage{config/5_profiles}
  \setclass{FS0210}
  \setcycle{2024-II}
  \setgroup{02}
  \usepackage{config/6_headers}
  %==<Definir uso de indice y glosarios >======================================
  % \pagestyle{uplain}
  % \setbool{test}{true}
  % \setbool{solutions}{true}
  % \makeindex
  % \makeglossaries
  % \glstoctrue
  %==<Definir glosarios y siglas>==============================================
  % %! TeX root = ../main.tex
%%%%%%%%%%%%%%%%%%%%%%%%%%%%%%%%%%%%%%%%%%%%%%%%%%%%%%%%%%%%%%%%%%%%
%% glossary.tex
%%
%% Glossary definition
% \newglossaryentry{}{ %label
%   name={}, % Name that appears on Glossary
%   description={}, % Entry description
%   sort={}, % [opt] If 'name' isn't right for sorting the entry on  Glossary
%   symbol={}, % [opt] If want to add an specific symbol for the entry
%              % [use] \glssymbol{label}
%   text={}, % [opt] If 'name' isn't right for the text body
%   plural={}, % [opt] Default plural $text+'s', if needed an other plural form
%              % [use] \glspl{label}
%   type={} % 'glossary-label', default='main'
% }
%%%%%%%%%%%%%%%%%%%%%%%%%%%%%%%%%%%%%%%%%%%%%%%%%%%%%%%%%%%%%%%%%%%%


  % %!TEX root = ../main.tex
%%%%%%%%%%%%%%%%%%%%%%%%%%%%%%%%%%%%%%%%%%%%%%%%%%%%%%%%%%%%%%%%%%%%
%% 01-Acronyms.tex
%% Acronyms definition
% \newacronym{label}{short}{long}
%%%%%%%%%%%%%%%%%%%%%%%%%%%%%%%%%%%%%%%%%%%%%%%%%%%%%%%%%%%%%%%%%%%%

%%%%%%%%%%%%%%%%%%%%%%%%%%%%%%%%%%%%%%%%%%%%%%%%%%%%%%%%%%%%%%%%%%%%%%%%%%%%%%%
%==<Inicia documento>==========================================================
%%%%%%%%%%%%%%%%%%%%%%%%%%%%%%%%%%%%%%%%%%%%%%%%%%%%%%%%%%%%%%%%%%%%%%%%%%%%%%%
\begin{document} %%%%%%%%%%%%%%%%%%%%%%%%%%%%%%%%%%%%%%%%%%%%%%%%%%%%%%%%%%%%%%
%%%%%%%%%%%%%%%%%%%%%%%%%%%%%%%%%%%%%%%%%%%%%%%%%%%%%%%%%%%%%%%%%%%%%%%%%%%%%%%
\thispagestyle{partial}
\begin{center}
  % {\Huge \thetitle} \\[2em]
  {\large \thepartial} \hfill
  {\large \thecycle} % \medskip \\
\end{center}
%%%%%%%%%%%%%%%%%%%%%%%%%%%%%%%%%%%%%%%%%%%%%%%%%%%%%%%%%%%%%%%%%%%%%%%%%%%%%%%
\ifthenelse{\boolean{solutions}}{
  \pdfbookmark{Índice General}{section}
  \tableofcontents
  \newpage\phantomsection
  \addcontentsline{toc}{section}{\thegroup  - \thecycle}
  \begin{center}
    \large{\thegroup}
    \hfill
    \large{\thecycle}
  \end{center}
}{
  \paragraph{Indicaciones}
  Lea y responda cada enunciado según se le indica. 
  Solo se atenderán dudas sobre la forma en que están redactas las preguntas, 
  por lo cual se le pide que se abstenga de hacer preguntas por sus 
  procedimientos.
  Escriba todas sus soluciones en hojas aparte (puede usar hojas en blanco
  engrapada o un cuaderno de examen).
  Asegúrese de escribir todos los pasos que realizó para obtener los
  resultados.
  Utilice lapicero de tinta azul o negra, si deja el examen en lápiz no se
  aceptarán reclamos posteriormente.
  Cuenta con 2 horas para realizar el examen.
  % A continuación se le ofrecen dos ejercicios por tema, debe escoger uno de los
  % dos y resolverlo. Solo se revisará un ejercicio por tema, así que indique 
  % claramente cual ejercicio está resolviendo. En caso de que no se indique 
  % claramente se tomará el primer ejercicio de un tema que se resuelva como el 
  % ejercicio a revisar. Escriba en el cuaderno de examen el número de pregunta, 
  % todos los datos, ecuaciones y procedimientos necesarios para obtener el 
  % resultado solicitado. No olvide que la respuesta es una frase final en la cual 
  % se responde a la pregunta del enunciado.
}
%%%%%%%%%%%%%%%%%%%%%%%%%%%%%%%%%%%%%%%%%%%%%%%%%%%%%%%%%%%%%%%%%%%%%%%%%%%%%%%
\begin{enumerate}[1.]
    \ifthenelse{\boolean{main}}{
  \exa[CH]{EEE} % \cite{book}
}{
  % \exa{}
}
\question{

}{
  % \begin{center}
  %   \includegraphics[width=0.3\textwidth]{}
  % \end{center}
}

\begin{enumerate}[a)]
  \qpart{
    \pts{}
  }{}
  \qpart{
    \pts{}
  }{
    \ptsdistro
    \begin{enumerate}[a)]
      \item .
      \begin{enumerate}
        \item .
          \pts{}
      \end{enumerate}
      \item .
      \begin{enumerate}
        \item .
          \pts{}
      \end{enumerate}
    \end{enumerate}
  }
\end{enumerate}

    \ifbool{solutions}{\newpage\phantomsection}{}
    \ifthenelse{\boolean{main}}{
  \exa[CH]{EEE} % \cite{book}
}{
  % \exa{}
}
\question{

}{
  % \begin{center}
  %   \includegraphics[width=0.3\textwidth]{}
  % \end{center}
}

\begin{enumerate}[a)]
  \qpart{
    \pts{}
  }{}
  \qpart{
    \pts{}
  }{
    \ptsdistro
    \begin{enumerate}[a)]
      \item .
      \begin{enumerate}
        \item .
          \pts{}
      \end{enumerate}
      \item .
      \begin{enumerate}
        \item .
          \pts{}
      \end{enumerate}
    \end{enumerate}
  }
\end{enumerate}

    \ifbool{solutions}{\newpage\phantomsection}{}
    \ifthenelse{\boolean{main}}{
  \exa[CH]{EEE} % \cite{book}
}{
  % \exa{}
}
\question{

}{
  % \begin{center}
  %   \includegraphics[width=0.3\textwidth]{}
  % \end{center}
}

\begin{enumerate}[a)]
  \qpart{
    \pts{}
  }{}
  \qpart{
    \pts{}
  }{
    \ptsdistro
    \begin{enumerate}[a)]
      \item .
      \begin{enumerate}
        \item .
          \pts{}
      \end{enumerate}
      \item .
      \begin{enumerate}
        \item .
          \pts{}
      \end{enumerate}
    \end{enumerate}
  }
\end{enumerate}

\end{enumerate}
%%%%%%%%%%%%%%%%%%%%%%%%%%%%%%%%%%%%%%%%%%%%%%%%%%%%%%%%%%%%%%%%%%%%%%%%%%%%%%%
% \ifthenelse{\boolean{solutions}}{}{
%   \begin{figure}[h!]
%     \centering
%     \begin{subfigure}{0.3\textwidth}
%       \includegraphics[width=\hsize]{}
%       \caption*{Problema 1}
%     \end{subfigure}
%     \hfill
%     \begin{subfigure}{0.3\textwidth}
%       \includegraphics[width=\hsize]{}
%       \caption*{Problema 2}
%     \end{subfigure}
%     \hfill
%     \begin{subfigure}{0.3\textwidth}
%       % \centering
%       \def\svgwidth{\hsize}
%       \import{figures/}{LenteConvergente.pdf_tex}
%       \caption*{Problema 3}
%     \end{subfigure}
%   \end{figure}
% }
%%%%%%%%%%%%%%%%%%%%%%%%%%%%%%%%%%%%%%%%%%%%%%%%%%%%%%%%%%%%%%%%%%%%%%%%%%%%%%%
% \newpage
% \phantomsection
% \begin{multicols}{2}
% \input{Formulario4.tex}
% \end{multicols}
% \includepdf[
%   addtotoc={
%     1, 
%     section,
%     1,
%     {Fórmulas del \thepartial - profesor (a)},
%     formulario
%   }
%   ]{formulario-P\thepartialNumber.pdf}
%%%%%%%%%%%%%%%%%%%%%%%%%%%%%%%%%%%%%%%%%%%%%%%%%%%%%%%%%%%%%%%%%%%%%%%%%%%%%%%
\end{document}%%%%%%%%%%%%%%%%%%%%%%%%%%%%%%%%%%%%%%%%%%%%%%%%%%%%%%%%%%%%%%%%%
%%%%%%%%%%%%%%%%%%%%%%%%%%%%%%%%%%%%%%%%%%%%%%%%%%%%%%%%%%%%%%%%%%%%%%%%%%%%%%%
